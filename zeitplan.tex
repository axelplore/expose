% !TeX root=main.tex

\section{Zeitplanung der Thesis}

Für den Ablauf der Thesis wurde ein Zeitplan erstellt. Der Zeitplan wird in Form eines Gantt-Diagramms\footnote{Vgl. \cite{Maylor2001}, S. 92-100} (Abbildung \ref{fig:zeitplanung}) zur Visualisierung der geplanten Arbeitsschritte dargestellt. Die Thesis beginnt in der Kalenderwoche 1 und endet in der Kalenderwoche 12. Die Arbeitsschritte sind in einzelne Phasen unterteilt. Die farblichen Balken beschreiben die jeweilige Phasendauer in Kalenderwochen. Ein genauer Beginn der Bearbeitung der Thesis ist noch nicht festgelegt.


\begin{figure}[H]
    \centering
    \caption{Gantt-Diagramm der Thesis-Zeitplanung}
    \label{fig:zeitplanung}
    \begin{ganttchart}[
        hgrid,
        vgrid,
        x unit=0.7cm, % Breite jeder Zeiteinheit
        title height=1,
        title label font=\bfseries\footnotesize,
        group right shift=0,
        group top shift=0.7,
        group height=.3,
        bar height=0.6,
        bar/.style={fill=cyan},
        incomplete/.style={fill=white},
        progress label text={},
        bar label font=\normalsize\rmfamily,
        milestone label font=\normalsize\rmfamily,
        milestone height=.8,
        milestone top shift=.1,
        milestone left shift=.1,
        milestone right shift=-.1
    ]{1}{12}
      \gantttitle{Kalenderwoche}{12} \\
      \gantttitlelist{1,...,12}{1} \\
    
      \ganttbar[bar/.append style={fill={rgb,255:red,100; green,180; blue,255}}]{Allgemeiner Überblick}{1}{2} \\
      \ganttbar[bar/.append style={fill={rgb,255:red,255; green,100; blue,100}}]{Literaturrecherche}{1}{6} \\
      \ganttbar[bar/.append style={fill={rgb,255:red,100; green,255; blue,100}}]{Aktueller Forschungsstand}{1}{3} \\
      \ganttbar[bar/.append style={fill={rgb,255:red,255; green,255; blue,100}}]{Fallstudie (DSR)}{3}{8} \\
      \ganttbar[bar/.append style={fill={rgb,255:red,255; green,180; blue,100}}]{Empirische Untersuchung}{6}{9} \\
      \ganttbar[bar/.append style={fill={rgb,255:red,180; green,100; blue,255}}]{Schreibphase}{9}{11} \\
      \ganttbar[bar/.append style={fill={rgb,255:red,255; green,100; blue,255}}]{Korrekturphase}{11}{12} \\
      \ganttbar[bar/.append style={fill={rgb,255:red,100; green,180; blue,255}}]{Fertigstellung und Abgabe}{12}{12}
    
    \end{ganttchart}
\end{figure}
    

\clearpage