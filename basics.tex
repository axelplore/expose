
% !TeX root=main.tex

\section{Grundlagen}

\subsection{Fundamentales}
Am Anfang war das Wort, siehe~\ref{sec:ziel}%es wird Bezug
. Und außerdem war es ziemlich finster\footnote{\cite{Mitt95:Latex}}.


\begin{table}[htb]
\caption{Preisliste}
\label{tab:preis}
\begin{center}
\begin{tabular}{c|l|r} % | für Linien,   mit Feldbegrenzung{c|p{60mm}|r}
Pos. & Beschreibung & Preis \\
\hline 			%Waagerechte Linie
1 & Prozessor & 198,98 \\
2 & Hauptspeicher & 87,66 \\
3 & Festplate & 128,88 \\
\hline
\multicolumn{2}{r}{Summe} & 1427,95 \\
\hline
\end{tabular}
\end{center}
\end{table}


schon Pythagoras \cite[S. 48ff]{Mitt95:Latex} wusste über das Dreieck, in der unten beschriebenen Formel~\ref{eq:wurzel}:
\begin{equation*}
a^2 + b^2 = c^2
\end{equation*}

\begin{align}
a^2 + 3\,b &=\cos \omega t \notag \\
\sqrt[3]{x^2 + y^2} &=7\,\ln \dfrac{5 + b}{7 - c} \label{eq:wurzel}
\end{align}


\begin{equation}
\left(
\begin{array}{cccc}
a_{1,1} & a_{1,2} & \cdots & a_{1,n} \\
a_{2,1} & a_{2,2} & \cdots & a_{2,n} \\
	\vdots & & \ddots & \vdots \\
a_{m,1} & a_{m,2} & \cdots & a_{m,n} \\
\end{array}
\right)
\end{equation}

\begin{align}
\sum_{i=1}^n &=n \cdot \dfrac{n+1}{2} \\
\text{e}^{\text{i}\,\pi} + 1 & = 0  % konstanten werden nicht kursiv gedruckt
\end{align}

Die Verfügbarkeit $a$ ergibt sich aus der Betriebszeit $t_{b}$ und der Geamtzeit $t_{ges}$ als
\begin{align}
availability &= \frac{operation time}{total time}  \\
&= \frac{8765 \text{h}}{8964 h} = 0,987\\
a &= \frac{t_b}{t_{ges}}
\end{align}

Sie beträgt $a = 0{,}98$ oder entsprechend 98\%. % {,} wichtig um die amerikanische Formatierung zu vermeiden

Die Fläche beträgt $A = 37$~m$^2$

\clearpage