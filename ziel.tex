% !TeX root=main.tex

\section{Forschungsfragen und Ziel der Arbeit}

KI wird in vielen Branchen die Prozesse, Abläufe und den Alltag von Menschen grundlegend verändern. KI-Technologien sind in Form von Chatbots in die breite Gesellschaft getragen worden und dienen als persönlicher Assistent von Büroarbeiten. Obwohl KI-Technologien im Alltag von Menschen präsenter werden, ist die spezifische Rolle und der Nutzen von KI in großen Unternehmen und Organisationen noch nicht umfassend erforscht. Vor allem die Unternehmen, wie beispielsweise große Schuheinzelhandelsketten, die sich nicht auf IT-Technologien spezialisiert haben, stehen vor der großen Herausforderungen KI-Technologien in ihre Geschäftsprozesse zu integrieren. Die zentrale Forschungsfrage dieser Arbeit lautet daher: Welche Rolle spielt die künstliche Intelligenz in großen Unternehmen, explizit in Schuheinzelhandelsketten, und welche konkreten Vorteile ergeben sich daraus?

Ziel der Arbeit ist, die Bedeutung und den Nutzen von KI im Schuheinzelhandel zu analysieren und konkrete Anwendungsszenarien aufzuzeigen. Dabei sollen insbesondere die Auswirkungen auf betriebliche Abläufe und einhergendenen Änderungen der Arbeitsprozesse der Mitarbeiter untersucht werden. Die Arbeit soll einen Beitrag dazu leisten, die Potenziale von KI im Schuheinzelhandel aufzuzeigen und die Unternehmen bei der Integration von KI-Technologien in ihre Geschäftsprozesse zu unterstützen. Gleichzeit soll die Arbeit dazu beitragen, die Mitarbeiter auf die Veränderungen vorzubereiten und die Akzeptanz von KI-Technologien zu fördern.

\clearpage