% !TeX root=main.tex

\section{Forschungsfragen und Ziel der Arbeit}

Die zentrale Forschungsfrage dieser Bacheloareit lautet:

\begin{description}
    \item[MRQ:] ''Wie können LLMs innerhalb des BI-Kontexts genutzt werden, um manuelle Prozesse zu automatisieren, und welche Auswirkungen hat dies auf Effizienz und Produktivität?''
\end{description}
Diese Frage bildet den Ausgangspunkt der Untersuchung und fokussiert auf die Integration von Large Language Models (LLMs) wie Microsoft 365 Copilot in Business Intelligence (BI)-Systemen, um manuelle Prozesse zu automatisieren.

Aus der zentralen Forschungsfrage ergeben sich folgende spezifische Forschungsfragen:
\begin{description}
    \item[RQ1:] Es sollen die spezifischen Prozesse, die am meisten von einer Automatisierung profitieren würden, identifiziert werden. Hierfür soll die Fragestellung \textbf{''Welche manuellen Prozesse innerhalb der BI-Landschaft eines Schuhhandelsunternehmens sind am zeitaufwendigsten und anfälligsten für Fehler?''} beantwortet werden.
    \item[RQ2:] Es sollen die technischen und praktischen Schritte untersucht werden, die erforderlich sind, um LLMs in die bestehenden BI-Prozesse zu integrieren. Für diesen Fall soll die Fragestellung \textbf{''Wie kann Microsoft 365 Copilot zur Automatisierung dieser identifizierten Prozesse implementiert werden?''} beantwortet werden.
    \item[RQ3:] Es soll die Leistung der automatisierten Prozesse im Vergleich zu den bisherigen manuellen Verfahren analysiert werden. Hierfür soll die Fragestellung \textbf{''Welche Auswirkungen hat die Automatisierung der Prozesse auf die Effizienz und Produktivität des Unternehmens?''} beantwortet werden. 
    \item[RQ4:] Es sollen potenzielle Schwierigkeiten und Herausforderungen bei der Implementierung von LLMs in BI-Systemen identifiziert und Lösungsansätze vorgeschlagen werden. Hierfür soll die Fragestellung \textbf{''Welche Herausforderungen und Grenzen können bei der Implementierung von Microsoft 365 Copilot in die BI-Landschaft eines Schuhhandelsunternehmens auftreten und wie können diese überwunden werden?''} beantwortet werden. 
\end{description}

Das Ziel dieser Bachelorarbeit ist es, ein fundiertes Verständnis dafür zu entwickeln, wie LLMs innerhalb von BI-Lösungen genutzt werden können, um manuelle Prozesse zu automatisieren und somit die Effizienz und Produktivität in einem Schuhhandelsunternehmen zu steigern. Durch die Identifikation spezifischer manueller Prozesse und die Implementierung von Microsoft 365 Copilot soll aufgezeigt werden, wie LLMs praktisch eingesetzt werden können, um die Datenanalyse und Berichtserstellung zu verbessern.

Darüber hinaus soll die Arbeit die Auswirkungen der Automatisierung auf die Genauigkeit der Datenverarbeitung und die Produktivität der Mitarbeiter untersuchen, um fundierte Empfehlungen für die weitere Integration von LLMs in BI-Systeme zu geben. Durch die Auseinandersetzung mit den Herausforderungen und Grenzen der Implementierung wird zudem ein umfassender Überblick über die praktischen und theoretischen Implikationen der Nutzung von LLMs im BI-Kontext vermittelt.

\clearpage