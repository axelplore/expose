% !TeX root=main.tex

\section{Vorläufige Gliederung der Bachelor-Thesis}

Die Thesis wurde in 7 Kapitel unterteilt und wurde insgesamt für eine ungefähre Seitenanzahl von 40-45 Seiten geplant. Die Gliederung der Thesis ist in Tabelle \ref{tab:gliederung} dargestellt.

\begin{longtable}{|p{9cm}|p{2.5cm}|}
    \caption{Gliederung der Bachelor-Thesis} \label{tab:gliederung} \\
    \hline
    \textbf{Kapitel} & \textbf{Seitenanzahl} \\
    \hline
    I.      Abbildungsverzeichnis & \\
    \hline
    II.     Abkürzungsverzeichnis & \\
    \hline
    III.    Formelverzeichnis & \\
    \hline
    IV.     Tabellenverzeichnis & \\
    \hline
    \textbf{1. Einleitung} & \textbf{4 Seiten} \\
    \hline
        1.1. Motivation & 1 Seite \\
        \hline
        1.2. Problemstellung und Zielsetzung & 1 Seite \\
        \hline
        1.3. Forschungsfragen und Methodik & 1 Seite \\
        \hline
        1.4. Aufbau der Arbeit & 1 Seite \\
    \hline
    \textbf{2. Theoretische Grundlagen} & \textbf{7 Seiten} \\
    \hline
        2.1. Grundlagen von Business Intelligence & 2 Seite \\
        \hline
        2.2. Einführung in Large Language Models & 2 Seiten \\
        \hline
        2.3. Relevanz und Potential von LLM in BI & 2 Seiten \\
        \hline
        2.4. Überblick über Microsoft 255 Copilot und änliche Technologien & 1 Seite \\
    \hline
    \textbf{3. Aktueller Forschungsstand} & \textbf{5 Seiten} \\
    \hline
        3.1. Systematische Literaturrecherche & 2 Seiten \\
        \hline
        3.2. Darstellung relevanter wissenschaftlicher Arbeiten & 2 Seiten \\
        \hline
        3.3. Aktuelle Herausforderungen und Forschungslücken & 1 Seite \\
    \hline
    \textbf{4. Fallstudienanalyse} & \textbf{10 Seiten} \\
    \hline
        5.1. Beschreibung des Unternehmens und der Abteilung & 1 Seite \\
        \hline
        5.2. Problemidentifikation & 2 Seiten \\
        \hline
        5.3. Entwicklung Modell zur Implementierung & 3 Seiten \\
        \hline
        5.4. Implementierung von Microsoft 255 Copilot & 3 Seiten \\
        \hline
        5.5. Dokumentation & 1 Seite \\
    \hline
    \textbf{5. Empirische Untersuchung} & \textbf{11 Seiten} \\
    \hline
        5.1. Entwicklung der Fragen \& Auswahl der Experten & 2 Seiten \\
        \hline
        5.2. Bildung \& systematisieren von Kategorien & 2 Seite \\
        \hline
        5.3. Durchführung der Expertbefragung & 2 Seiten \\
        \hline
        5.4. Zuordnung der Textpassagen zu den Kategorien & 2 Seiten \\
        \hline
        5.5. Analyse der Häufigkeit und Verteilung der Kategorien & 2 Seiten \\
        \hline
        5.6. Interpretation der Ergebnisse im Kontext der Forschungsfragen & 1 Seite \\
    \hline
    \textbf{6. Diskussion der Ergebnisse \& Ausblick} & \textbf{4 Seiten} \\
    \hline
        6.1. Zusammenfassen der wesentlichen Ergebnisse & 1 Seite \\
        \hline
        6.2. Diskussion der Auswirkungen auf Effizienz, Genauigkeit und Produktivität & 1 Seite \\
        \hline
        6.3. Bewertung der Herausforderungen und Grenzen der Implementierung & 1 Seite \\
        \hline
        8.4. Ausblick auf weiterführende Forschung & 1 Seite \\
    \hline
    \textbf{7. Fazit und Ausblick} & \textbf{1 Seite} \\
    \hline
    V. Literaturverzeichnis & \\
    \hline
    VI. Anhang & \\
    \hline
\end{longtable}



\clearpage