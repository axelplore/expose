% !TeX root=main.tex

\section{Vorläufige Gliederung der Bachelor-Thesis}


\begin{longtable}{|p{6cm}|p{6cm}|}
    \hline
    \textbf{Kapitel} & \textbf{Seitenanzahl} \\
    \hline
    I.      Abbildungsverzeichnis & \\
    \hline
    II.     Abkürzungsverzeichnis & \\
    \hline
    III.    Formelverzeichnis & \\
    \hline
    IV.     Tabellenverzeichnis & \\
    \hline
    1.  Abstract & \textbf{1 Seite} \\
    \hline
    2.  Einleitung & \textbf{5 Seiten} \\
    \hline
        2.1. Motivation & 1 Seite \\
        \hline
        2.2. Problemstellung und Zielsetzung & 1 Seite \\
        \hline
        2.3. Forschungsfragen und Methodik & 2 Seite \\
        \hline
        2.4. Aufbau der Arbeit & 1 Seite \\
    \hline
    3.  Theoretische Grundlagen & \textbf{7 Seiten} \\
    \hline
        3.1. Grundlagen von Business Intelligence & 1 Seite \\
        \hline
        3.2. Einführung in Large Language Models & 2 Seiten \\
        \hline
        3.3. Relevanz und Potential von LLM in BI & 2 Seiten \\
        \hline
        3.4. Überblick über Microsoft 365 Copilot und änliche Technologien & 2 Seiten \\
    \hline
    4. Aktueller Forschungsstand & \textbf{6 Seiten} \\
    \hline
        4.1. Systematische Literaturrecherche & 2 Seiten \\
        \hline
        4.2. Darstellung relevanter wissenschaftlicher Arbeiten & 2 Seiten \\
        \hline
        4.3. Aktuelle Herausforderungen und Forschungslücken & 2 Seiten \\
    \hline
    5. Fallstudienanalyse & \textbf{8 Seiten} \\
    \hline
        5.1. Beschreibung des Unternehmens und der Abteilung & 2 Seiten \\
        \hline
        5.2. Identifikation und Beschreibung manueller Prozesse & 2 Seiten \\
        \hline
        5.3. Implementierung von Microsoft 365 Copilot & 2 Seiten \\
        \hline
        5.4. Dokumentation & 2 Seiten \\
    \hline
    6. Empirische Untersuchung & \textbf{8 Seiten} \\
    \hline
        6.1. Datenerhebung vor und nach der Implementierung & 2 Seiten \\
        \hline
        6.2. Quantitative und qualitative Datenanalyse & 2 Seiten \\
        \hline
        6.3. Vergleich der Ergebnisse & 2 Seiten \\
        \hline
        6.4. interpretation der Ergebnisse im Kontext der Forschungsfragen & 2 Seiten \\
    \hline
    7. Diskussion der Ergebnisse& \textbf{5 Seiten} \\
    \hline
        7.1. Zusammenfassen der wesentlichen Ergebnisse & 2 Seiten \\
        \hline
        7.2. Diskussion der Auswirkungen auf Effizienz, Genauigkeit und Produktivität & 2 Seiten \\
        \hline
        7.3. Bewertung der Herausforderungen und Grenzen der Implementierung & 1 Seite \\
    \hline
    8. Fazit und Ausblick & \textbf{3 Seiten} \\
    \hline
        8.1. Zusammenfassung der Arbeit und der wichtigsten Erkenntnisse & 1 Seite \\
        \hline
        8.2. Implikationen für die Praxis & 1 Seite \\
        \hline
        8.3. Ausblick auf weiterführende Forschung & 1 Seite \\
    \hline
    V. Literaturverzeichnis & \\
    \hline
    VI. Anhang & \\
    \hline
    \caption{Gliederung der Bachelor-Thesis} \label{tab:gliederung} \\
\end{longtable}



\clearpage