
% !TeX root = main.tex

\section{Einleitung} 
\label{sec:einleitung}

Die KI hat in den letzten Jahren eine rasante Entwicklung durchlaufen. Die Studie von Dwivedi et al.\footnote{Vgl. \cite{DwivediHughes2021}} beschreibt KI als eine transformative Technologie, die menschliche Aufgaben und Aktivitäten in vielen Bereichen ergänzen oder ersetzen kann. Die rasanten Fortschritte in algorithmischem maschinellem Lernen und autonomen Entscheidungsprozessen schaffen neue Möglichkeiten für Innovationen. KI hat das Potenzial, menschliche Aufgaben in Bereichen wie Finanzen, Gesundheitswesen, Fertigung, Einzelhandel, Lieferkette, Logistik und Versorgungsunternehmen zu erweitern oder zu ersetzen.\footnote{Vgl. \cite{DwivediHughes2021}, S. 1} 
In dieser Arbeit konzentriert sich die Analyse auf die Bedeutung und den Nutzen von KI im Schuheinzelhandel.

Im Jahre 2005 stellte Buchanan\footnote{Vgl. \cite{Buchanan2005}} einige historische Schlüsselmomente der KI-Entwicklung zusammen. Er verweist auf John McCarthys Schrift ''Programs with Common Sense'' aus dem Jahr 1958, in der McCarthy sich für eine deklarative Wissensdarstellung einsetzt, die leicht zu manipulieren ist und die KI-Forschung maßgeblich beeinflusst hat, wodurch der Grundstein für die Künstliche Intelligenz gelegt wurde. Bereits zwei Jahre zuvor, 1956, hatte John McCarthy\footnote{Vgl. \cite{Mccarthy2006}} eine KI entwickelt und an einer Maschine geforscht, die wie ein Mensch denken, Probleme lösen und sich selbst verbessern konnte. Diese KI besitzt Schlüsselmerkmale wie adaptive Steuerung, bessere Handhabung und Wiederverwendbarkeit gespeicherten Wissens. Kahn's Werk\footnote{Vgl. \cite{Kahn2001}} beschreibt Dr. Shannons spätere Arbeiten an schachspielenden Maschinen und einer elektronischen Maus, die ein Labyrinth bewältigen konnte. Diese Arbeiten trugen maßgeblich zur Entstehung des Forschungsfeldes der Künstlichen Intelligenz bei, das sich der Entwicklung denkender Maschinen widmet. 

Mit der Veröffentlichung des NLP-Modells GPT-3\footnote{Vgl. \cite{Brown2020}} im Jahr 2020 hat OpenAI einen Meilenstein in der KI-Forschung gesetzt. GPT-3 ist ein Sprachmodell, das auf der Grundlage von 175 Milliarden Parametern trainiert wurde und in der Lage ist, menschenähnliche Texte zu generieren. Das Modell hat das Potenzial, die Art und Weise, wie wir mit Computern interagieren, zu revolutionieren und die KI-Technologie auf ein neues Niveau zu heben. OpenAI hat das Modell als Chatbot kostenlos und für jedermann zugänglich gemacht, wodurch die KI-Technologie in die Gesellschaft eingeführt wurde. 

Die Potenziale setzen Industrie und Unternehmen unter Druck, KI in deren Prozesse zu integrieren. Klarheit darüber, wie KI in Wirtschaftsmodelle integriert wird, ist entscheidend für genaue Prognosen und Entscheidungsfindung.\footnote{Vgl. \cite{Lu2021}} Das Forschungspapier von Lee et al.\footnote{Vgl. \cite{Lee2018}, S.21} bietet umfassende Einblicke in die grundlegenden Elemente und die Struktur von Künstlicher Intelligenz in der Industrie. Insbesondere die ''ABCDE''-Elemente (Analytics, Big Data, Cloud, Domain-Knowhow und Evidence) können als Grundlage dienen, um die Bedeutung und die Struktur von KI im Schuheinzelhandel zu analysieren und darzustellen. 

Floridi\footnote{Vgl. \cite{Floridi2019}, S.4-5} argumentiert, dass die Qualität der Daten entscheidend für den Erfolg von KI ist, und hebt Beispiele hervor, in denen gut kuratierte, qualitativ hochwertige Datensätze verwendet wurden, um beeindruckende Ergebnisse zu erzielen. Die Forschung von Wang et al.\footnote{Vgl. \cite{Wang2024}, S.4-12} zeigt, wie KI, insbesondere grafische konvolutionale neuronale Netzwerke und Meta-Learning, zur Segmentierung von Plantardruckbildern eingesetzt werden kann. KI ermöglicht die Extraktion von Design-Elementen aus komplexen Bilddatensätzen und somit die Anpassung von Schuhen an spezifische Gesundheitsanforderungen. Das Paper von Perez et al.\footnote{Vgl. \cite{Perez2018}, S.1-10} beleuchtet die Anwendung von KI, insbesondere durch Computervisionssysteme zur Objekterkennung und Trajektorien Planung mit Kollisionsvermeidung. Diese Technologien sind zentral für die Automatisierung und können auf den Schuhhandel übertragen werden, etwa zur Optimierung von Lagerhaltung und Bestandsverwaltung. Das Forschungspapier von Li et al.\footnote{Vgl. \cite{Shuyang2021}} zeigt, dass industrielle KI durch die Nutzung von Datenanalyse und maschinellem Lernen große Mengen an industriellen Daten (inkl. Marketing-, Einkaufs-, Fertigungs- und Lieferketten-Daten) in nützliche Erkenntnisse, Muster und Vorhersagen umwandelt, die automatisierte Entscheidungsfindung, Produktentwicklung und effektives Marketing ermöglichen können.

\newpage