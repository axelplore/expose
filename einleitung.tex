
% !TeX root = main.tex

\section{Einleitung} 
\label{sec:einleitung}

Die Studie von Dwivedi et al.\footnote{Vgl. \cite{DwivediHughes2021}, S.11-13} beschreibt KI als eine transformative Technologie, die menschliche Aufgaben und Aktivitäten in vielen Bereichen ergänzen oder ersetzen kann. Ein besonders vielversprechender Aspekt in diesem Bereich ist die Anwendung von Large Language Models (LLM) in Business Intelligence (BI)-Lösungen. Diese Integration zielt darauf ab, Arbeitsabläufe zu rationalisieren, menschliche Fehler zu reduzieren und datengesteuerte Entscheidungsprozesse zu ermöglichen. 
Die Einzelhandelsbranche, insbesondere der Schuhsektor, bietet vielseitige Gelegenheiten, diese Vorteile zu nutzen. Manuelle Prozesse im Einzelhandel, wie z. B. Bestandsverwaltung, Kundenservice und Verkaufsanalyse, sind oft zeitaufwändig und fehleranfällig.\footnote{Vgl. \cite{Perez2018}, S.1-10; \cite{Lee2018}, S.20} Durch den Einsatz von LLM in BI-Lösungen könnten diesen Aufgaben automatisiert werden, so dass Einblicke in Echtzeit möglich sind und sich die Mitarbeiter auf strategischere Tätigkeiten konzentrieren können.


Mit der Veröffentlichung des LLM GPT-3 im Jahr 2020 hat OpenAI einen Meilenstein in der KI-Forschung gesetzt. GPT-3 ist ein Sprachmodell, das auf der Grundlage von 175 Milliarden Parametern trainiert wurde und in der Lage ist, menschenähnliche Texte zu generieren.\footnote{Vgl. \cite{Brown2020}, S.1878} Das LLM hat das Potenzial, die Art und Weise, wie wir mit Computern interagieren, zu revolutionieren und die KI-Technologie auf ein neues Niveau zu heben.\footnote{Vgl. \cite{Lu2021}, S.1046-1047} OpenAI hat das Modell als Chatbot kostenlos und für jedermann zugänglich gemacht, wodurch die KI-Technologie in die Gesellschaft eingeführt wurde. Erste Studien haben das Potenzial von LLM für die Umgestaltung von Geschäftsprozessen aufgezeigt. So kann laut einer Studie von Prabhavathi et al.\footnote{Vgl. \cite{Prabhavathi2019}, S.161-162} die Integration von LLM im Einzelhandel das Kundenerlebnis durch die Automatisierung von Abfrageantworten und die Bereitstellung personalisierter Empfehlungen erheblich verbessern. In ähnlicher Weise zeigten Ibrahima et. al.\footnote{Vgl. \cite{Ibrahima2021}, S.33}, wie LLM-gestützte BI-Lösungen, die Genauigkeit der Bestandsverwaltung durch die Analyse von Verkaufsmustern und die Vorhersage des Lagerbedarfs verbessern können.

Jüngste Fortschritte unterstreichen die Relevanz dieser Integration weiter. Die Entwicklung von Tools wie Microsofts Copilot für Microsoft 365 zeigt die praktische Anwendung von LLMs bei der Automatisierung von Routineaufgaben und der Steigerung der Produktivität im Unternehmenskontext. Das Microsoft Copilot-System nutzt LLMs, um Benutzer bei der Erstellung von Inhalten, der Analyse von Daten und der Automatisierung sich wiederholender Aufgaben zu unterstützen und so den manuellen Aufwand in BI-Prozessen erheblich zu reduzieren.\footnote{Vgl. \cite{Spataro2024}, S.1-7}

Die Relevanz dieses Themas wird durch die steigende Nachfrage nach effizienten Datenmanagement- und Analysewerkzeugen in der heutigen datengesteuerten Geschäftsumgebung unterstrichen.\footnote{Vgl. \cite{Syam2018}, S.135-136} Durch die Untersuchung des Potenzials von LLMs innerhalb von BI-Lösungen soll diese Arbeit einen Einblick geben, wie diese fortschrittlichen Technologien genutzt werden können, um Geschäftsprozesse zu automatisieren und zu optimieren.

\newpage