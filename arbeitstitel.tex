
% !TeX root=main.tex

\section{Arbeitstitel}

Der Arbeitstitel für diese Bachelorarbeit lautet: "Automatisierung manueller Prozesse mit LLMs innerhalb von BI-Lösungen". Dieser Titel wurde gewählt, um die zentrale Fragestellung der Arbeit prägnant und klar zu kommunizieren. Im Fokus stehen dabei die Nutzung von Large Language Models (LLMs) zur Automatisierung verschiedener manueller Prozesse innerhalb von Business Intelligence (BI)-Lösungen und die damit verbundenen Auswirkungen auf Effizienz, Genauigkeit und Produktivität.

Der Titel wurde nach sorgfältiger Überlegung entwickelt, um sowohl die technischen als auch die anwendungsbezogenen Aspekte der Arbeit zu berücksichtigen. Der Ausgangspunkt war das breite Feld der Künstlichen Intelligenz (AI) in Verbindung mit BI. Um die Arbeit jedoch spezifischer und fokussierter zu gestalten, wurde der Begriff ''AI'', erst durch ''NLP'' und schließlich durch ''LLMs'' ersetzt, da diese spezifischen Modelle besonders vielversprechende Anwendungsmöglichkeiten in der Automatisierung bieten.

Darüber hinaus wurde der Titel so formuliert, dass er die Implementierung und Evaluierung von LLMs in einem praktischen Geschäftsumfeld (hier: ein Schuhhandelsunternehmen) impliziert. Dies unterstreicht die praktische Relevanz und den Innovationscharakter der Untersuchung.

Um sicherzustellen, dass der gewählte Titel die Intention und den Inhalt der Arbeit optimal widerspiegelt, wurden verschiedene alternative Titel in Betracht gezogen:

\begin{enumerate}
    \item Die ersten Überlegungen zielten darauf ab, den Fokus auf den Schuheinzelhandel und die Anwendungsszenarien von KI zu legen:
    
    ''Die Rolle von künstlicher Intelligenz im Schuheinzelhandel: Bedeutung, Nutzen und Anwendungsszenarien''

    \item Um einen konkreteren Ansatz zu haben, wurde der Titel auf die Automatisierung manueller Prozesse mit KI in BI-Lösungen ausgerichtet:
    
    ''Automatisierung manueller Prozesse mit KI und BI-Lösungen in einem Schuhhandelsunternehmen''

    \item Eine weitere Alternative war es, den Schwerpunkt auf die Anwendung auf NLPs in manuellen BI-Prozessen zu legen:
    
    ''Automatisierte Datenanalyse und Berichtserstellung mit NLPS in einem Schuhhandelsunternehmen''

    \item Weitere Alternativen für die Konzentration auf die Anwendung von LLMs in BI-Prozessen waren:
    
    ''Einsatz von Large Language Models zur Automatisierung von Geschäftsprozessen im BI-Kontext''

    ''Optimierung der BI-Landschaft durch Automatisierung manueller Prozesse mit LLMs''
\end{enumerate}
''Einsatz von Large Language Models zur Automatisierung von Geschäftsprozessen im BI-Kontext''

\clearpage